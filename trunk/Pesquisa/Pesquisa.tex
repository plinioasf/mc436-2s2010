%-%-%-%-%-%-%-%-%-%-%-%-%-%-%-%-%-%-%-%-%-%-%-%-%-%
% MC436 Engenharia de Software                    %  
% Projeto : Sistema de gerenciamento              %
%           de conferências técnicas              %
% Data:05/08/2010                                 %
% Unicamp,Campinas,São Paulo,Brasil               % 
% Grupo:                                          %
%       - Mariana Mendes Caspirro                 % 
%       - Plinio Augusto Soriano Freire           %
%       - Tiago Chedraoui Silva                   % 
%-%-%-%-%-%-%-%-%-%-%-%-%-%-%-%-%-%-%-%-%-%-%-%-%-%
\documentclass[letter]{article}


%%% fontes %%%
\usepackage[T1]{fontenc}
\usepackage[brazil]{babel}    % dá suporte para os termos na língua portuguesa do Brasi
\usepackage[utf8]{inputenc}   % acentuação

%%% outros %%%
\usepackage{textcomp}
\usepackage{color}       
\usepackage{indentfirst} %retira padrao americano de paragrafos
\usepackage{multicol}   
\usepackage[linkbordercolor={1 1 1},urlcolor=black,colorlinks=true]{hyperref} % links

% Capa estilizada %
\newcommand*{\titleTMB}{\begingroup \centering \settowidth{\unitlength}{\LARGE MC 613} \vspace*{\baselineskip} {\large\scshape  Turma A}\\[\baselineskip] \rule{11.0cm}{1.6pt}\vspace*{-\baselineskip}\vspace*{2pt} \rule{11.0cm}{0.4pt}\\[\baselineskip] {\LARGE  Pesquisa de Sistemas de  }\\[0.2\baselineskip] {\LARGE  Gerenciamento de Conferências Tecnológicas }\\[0.2\baselineskip] {\itshape MC 436 - Engenharia de Software - Segundo Semestre de 2010}\\[0.2\baselineskip] \rule{11.0cm}{0.4pt}\vspace*{-\baselineskip}\vspace{3.2pt} \rule{11.0cm}{1.6pt}\\[\baselineskip] {\large\scshape Professora: Ariadne Maria Brito Rizzoni Carvalho}\par \vfill {\normalsize   \scshape 
    \begin{center} 
      \begin{tabular}{  l  l  p{5cm} } 
        Mariana Mendes Caspirro & RA: 082204\\
        Plinio Augusto Soriano Freire & RA:  082505\\
        Tiago Chedraoui Silva  & RA: 082941\\
      \end{tabular} \end{center}
    \itshape \today }\\[\baselineskip] \vspace{3.2pt} \endgroup}


\begin{document}
\titleTMB 
\newpage
\section{ V Conferência Sul-Americana em Ciência e Tecnologia}
O primeiro sistema analisado foi da V Conferência Sul-Americana em Ciência e Tecnologia aplicado ao Governo Eletrônico (V CONeGOV).\\ Link: \url{http://www.i3g.org.br/conegov/home/index.php}

\subsection{Pré-Conferência}

\begin{itemize}
\item Chamadas de trabalho
\item Programação do evento
\item Facilidades para Registro (Inscrição)
\item Local do evento
\item Citação dos documentos aprovados para o evento             
\end{itemize}

\subsection{Conferência}
\begin{itemize}
\item Possibilidade de ver o evento ao vivo

\item Apresentação dos tópicos de interesse

\item Feedback (Contato com o evento assim como a Secretaria Executiva e a Equipe organizadora)

\end{itemize}

\subsection{Pós-conferência}
\begin{itemize}
\item Nada
\end{itemize}


\subsection{Pontos fortes}
\begin{itemize}
\item Transmissão ao vivo do evento

\item Modelo para submissão de artigos

  
\end{itemize}


\subsection{Pontos fracos}
\begin{itemize}
\item Não há suporte para acomodação e transporte dos inscritos (como chegar, descontos para viagem)
\item Nenhuma informação pós-conferência
\end{itemize}

\section{ Anpei}
O segundo sistema analisado foi da Conferência Anpei (Associação Nacional de Pesquisa e Desenvolvimento das Empresas Inovadoras). Link: http://www.fozzy.com.br/anpei

\subsection{Pré-Conferência}

\begin{itemize}
\item Chamadas de trabalho
\item Programação do evento
\item Facilidades para Registro (Inscrição)
\item Local do evento
\item Indicação de como chegar assim como pacotes de viagem e hospedagem        \end{itemize}

\subsection{Conferência}
\begin{itemize}
\item Newsletter
\item Feedback (Contato com o evento)
\end{itemize}

\subsection{Pós-conferência}
\begin{itemize}
\item Galeria de Fotos
\end{itemize}


\subsection{Pontos fortes}
\begin{itemize}
\item Indicação de hospedagem e pacotes de viagem
\end{itemize}


\subsection{Pontos fracos}
\begin{itemize}
\item Poucas informações pós-conferência
\end{itemize}



\section{FISL}
O terceiro sistema analisado foi do Fórum Internacional Software Livre (FISL) é considerado o maior encontro de comunidades de software livre da América Latina e um dos maiores do mundo que se tornou um ponto de encontro anual, de pessoas de todos os lugares do Brasil e do mundo para debates técnicos e estratégicos sobre o desenvolvimento e o uso do Software Livre.


\subsection{Pré-conferência}
Informações fornecidas:
\begin{itemize}
\item Programação com sub-divisão por temas
\item Planta do evento
\item Roteiro de bares
\item Criação de Rede Social
\end{itemize}
Coleta de dados:
\begin{itemize}
\item Utilização do software Papers para avaliação, através do qual era permitido submeter suas palestras e avaliar outras. Possui sistema de pontuação configurável.  
\end{itemize}
\subsection{Conferência}
\begin{itemize}
\item Erratas - Utilização de televisões
\item Tv software livre
\item Rádio Software livre
\end{itemize}

\subsection{Pós-conferência}
\begin{itemize} 
\item Galeria de Fotos
\end{itemize}

\subsection{Pontos fortes}
\begin{itemize}
\item Programação com sub-divisão por temas
\item Sistema de avaliação completo
\end{itemize}

\subsection{Pontos fracos}
\begin{itemize}
\item Poucas informações pós-conferência
\end{itemize}

\end{document}
