%-%-%-%-%-%-%-%-%-%-%-%-%-%-%-%-%-%-%-%-%-%-%-%-%-%
% MC436 Engenharia de Software                    %  
% Projeto : Sistema de gerenciamento              %
% de conferências técnicas                        %
% Data:05/08/2010                                 %
% Unicamp,Campinas,São Paulo,Brasil               % 
% Grupo:                                          %
% - Mariana Mendes Caspirro                       % 
% - Plinio Augusto Soriano Freire                 %
% - Tiago Chedraoui Silva                         % 
%-%-%-%-%-%-%-%-%-%-%-%-%-%-%-%-%-%-%-%-%-%-%-%-%-%
\documentclass[letter]{article}


%%% fontes %%%
\usepackage[T1]{fontenc}
\usepackage[brazil]{babel}    % dá suporte para os termos na língua portuguesa do Brasi
\usepackage[utf8]{inputenc}   % acentuação

%%% outros %%%
\usepackage{textcomp}
\usepackage{color}       
\usepackage{indentfirst} %retira padrao americano de paragrafos
\usepackage{multicol}   
\usepackage[linkbordercolor={1 1 1},urlcolor=black,colorlinks=true]{hyperref} % links

% Capa estilizada %
\newcommand*{\titleTMB}{\begingroup \centering \settowidth{\unitlength}{\LARGE MC 613} \vspace*{\baselineskip} {\large\scshape  Turma A}\\[\baselineskip] \rule{11.0cm}{1.6pt}\vspace*{-\baselineskip}\vspace*{2pt} \rule{11.0cm}{0.4pt}\\[\baselineskip] {\LARGE  Pesquisa de Sistemas de  }\\[0.2\baselineskip] {\LARGE  Gerenciamento de Conferências Tecnológicas }\\[0.2\baselineskip] {\itshape MC 436 - Engenharia de Software - Segundo Semestre de 2010}\\[0.2\baselineskip] \rule{11.0cm}{0.4pt}\vspace*{-\baselineskip}\vspace{3.2pt} \rule{11.0cm}{1.6pt}\\[\baselineskip] {\large\scshape Professora: Ariadne Maria Brito Rizzoni Carvalho}\par \vfill {\normalsize   \scshape 
    \begin{center} 
      \begin{tabular}{  l  l  p{5cm} } 
        Mariana Mendes Caspirro & RA: 082204\\
        Plinio Augusto Soriano Freire & RA:  082505\\
        Tiago Chedraoui Silva  & RA: 082941\\
      \end{tabular} \end{center}
    \itshape \today }\\[\baselineskip] \vspace{3.2pt} \endgroup}


\begin{document}
\titleTMB 
\newpage

\section{Fozzy}
O primeiro sistema analisado foi da Conferência Anpei (Associação Nacional de Pesquisa e Desenvolvimento das Empresas Inovadoras). Link: http://www.fozzy.com.br/anpei

\subsection{Pré-Conferência}

\begin{itemize}
\item Chamadas de trabalho
\item Programação do evento
\item Facilidades para Registro (Inscrição)
\item Local do evento
\item Indicação de como chegar assim como pacotes de viagem e hospedagem        \end{itemize}

\subsection{Conferência}
\begin{itemize}
\item Newsletter
\item Feedback (Contato com o evento)
\end{itemize}

\subsection{Pós-conferência}
\begin{itemize}
\item Galeria de Fotos
\end{itemize}


\subsection{Pontos fortes}
\begin{itemize}
\item Indicação de hospedagem e pacotes de viagem
\end{itemize}


\subsection{Pontos fracos}
\begin{itemize}
\item Poucas informações pós-conferência
\end{itemize}



\section{FISL - Papers}
O segundo sistema analisado foi o papers, que é utilizado no Fórum Internacional Software Livre (FISL) e suporta:
\subsection{Pré-conferência}
\begin{itemize}
\item Submissão de proposta de palestra
\item Julgamento das propostas de palestra de acordo com sistema de pontuação configurável.
\item Inclusão de palestras convidadas/de patrocinadores (não submetidas)
\item Publica programação com sub-divisão por temas
\end{itemize}
\subsection{Conferência}
\begin{itemize}
\item Erratas - Utilização de televisões
\item Tv software livre
\item Rádio Software livre
\end{itemize}

\subsection{Pós-conferência}
\begin{itemize} 
\item Galeria de Fotos
\end{itemize}

\subsection{Pontos fortes}
\begin{itemize}
\item Programação com sub-divisão por temas
\item Sistema de avaliação completo
\item Idiomas do sistema espanhol, inglês e português do conteúdo.
\end{itemize}

\subsection{Pontos fracos}
\begin{itemize}
\item Poucas informações pós-conferência
\item Não há fornecimento de material apresentado ao público.
\end{itemize}

\section{Open Conference System}
\subsection{Sobre}
O terceiro sistema analisado é um “open source solution” para controlar e publicar conferências online que foi projetado para reduzir o tempo e a energia dedicados às tarefas de escritório e administrativas associadas com o controle de uma conferência ao melhorar a eficiência de processos editoriais. 


\subsection{Pré-Conferência}
Open Conference System permite:
\begin{itemize}
\item Criar um Web site da conferência
\item Controlar as conferências que ocorrem mais de uma vez (por exemplo, anual)
\item Compor e emitir uma chamada para papers
\item Aceitar eletronicamente as submissões de papers e abstracts
\item Permitir aos autores submeterem os papers e editar seu trabalho
\item Conduzir revisões paritárias
\item Afixar continuações e papers de conferência em um formato procurado
\item Programe uma conferência
\item Postar, se você deseja, as séries de dados originais
\item Registrar participantes, incluindo a aceitação de pagamentos
\item Integrar as discussões online
\item Utilizar um sistema padrão de email
\item Suportar diversas línguas 
\item Aproveitar-se de um código mais customizável, mais evolutivo e seguro 
\end{itemize}


\subsection{Conferência}
\begin{itemize}
\item Erratas - Correção on-line
\end{itemize}

\subsection{Pós-conferência}
\begin{itemize} 
\item Galeria de Fotos
\end{itemize}

\subsection{Pontos fortes}
\begin{itemize}
\item Sistema de avaliação completo
\item Suporte a diversos idiomas
\end{itemize}

\subsection{Pontos fracos}
\begin{itemize}
\item Poucas informações pós-conferência
\end{itemize}





\section{Easy Chair}
\subsection{Sobre}
O quarto sistema analisado foi o Easy Chair. Ele é um sistema grátis de gerenciamento de conferência que é flexível, fácil de usar, e tem muitas ferramentas para torna-lo adaptável para vários modelos de conferências. (Link de conferência que utiliza easy chair: http://www.adbis2009.org)


\subsection{Pré-Conferência}

\begin{itemize}
\item Inscrição dos participantes
\item Submissão de documentos/trabalhos
\item Datas importantes (programação)
\item Acomodações (preços, pacotes, facilidades)
\end{itemize}


\subsection{Conferência}
\begin{itemize}
\item Possibilidade de feedback
\item Novidades da conferência
\end{itemize}

\subsection{Pós-conferência}
\begin{itemize} 
\item Slides de apresentações
\item Galeria de fotos
\item Informações sobre publicações referentes a conferência
  
\end{itemize}

\subsection{Pontos fortes}
\begin{itemize}
\item Disponibilidade de material pós conferência
\item Informações sobre hospedagem e sobre as datas da conferência
\end{itemize}

\subsection{Pontos fracos}
\begin{itemize}
\item Má organização nas informações disponibilizadas no site
\end{itemize}





\end{document}
