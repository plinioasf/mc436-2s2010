%-%-%-%-%-%-%-%-%-%-%-%-%-%-%-%-%-%-%-%-%-%-%-%-%-%
% MC436 Engenharia de Software                    %  
% Projeto : Sistema de gerenciamento              %
% de conferências técnicas                        %
% Data:18/08/2010                                 %
% Unicamp,Campinas,São Paulo,Brasil               % 
% Grupo:                                          %
% - Mariana Mendes Caspirro                       % 
% - Murilo Fossa Vicentini                        % 
% - Plinio Augusto Soriano Freire                 %
% - Tiago Chedraoui Silva                         %
% - Raquel Mayume Kawamoto                        % 
%-%-%-%-%-%-%-%-%-%-%-%-%-%-%-%-%-%-%-%-%-%-%-%-%-%
\documentclass[letter]{article}

%%% fontes %%%
\usepackage[T1]{fontenc}
\usepackage[brazil]{babel}    % dá suporte para os termos na língua portuguesa do Brasi
\usepackage[utf8]{inputenc}   % acentuação
\usepackage{ae,aecompl,aeguill}       % pdfs mais bonitos =)

%%% outros %%%
\usepackage{multirow}
\usepackage{textcomp}
\usepackage{color}       
\usepackage{indentfirst}      % retira padrao americano de paragrafos
\usepackage{multicol}   
\usepackage[linkbordercolor={1 1 1},urlcolor=black,colorlinks=false]{hyperref} % links

\usepackage{indentfirst} %retira padrao americano de paragrafos
\usepackage{ae,aecompl}
\usepackage{pslatex}
\usepackage{epsfig}
\usepackage{verbatim}
\usepackage{pstricks}
\usepackage[letterpaper]{geometry}
\geometry{verbose,lmargin=3cm,rmargin=3cm}


% Capa estilizada %
\newcommand*{\titleTMB}{\begingroup \centering \settowidth{\unitlength}{\LARGE MC 613} \vspace*{\baselineskip} {\large\scshape  Turma A}\\[\baselineskip] \rule{11.0cm}{1.6pt}\vspace*{-\baselineskip}\vspace*{2pt} \rule{11.0cm}{0.4pt}\\[\baselineskip] {\LARGE Levantamento de requisitos para }\\[0.2\baselineskip] {\LARGE Sistema de gerenciamento de Conferências Tecnológicas }\\[0.2\baselineskip] {\itshape MC 436 - Engenharia de Software - Segundo Semestre de 2010}\\[0.2\baselineskip] \rule{11.0cm}{0.4pt}\vspace*{-\baselineskip}\vspace{3.2pt} \rule{11.0cm}{1.6pt}\\[\baselineskip] {\large\scshape Professora: Ariadne Maria Brito Rizzoni Carvalho}\par \vfill {\normalsize   \scshape 
    \begin{center} 
      \begin{tabular}{  l  l  p{5cm} } 
        Mariana Mendes Caspirro & RA: 082204\\
        Murilo Fossa Vicentini & RA: 082335 \\
        Plinio Augusto Soriano Freire & RA:  082505\\
        Tiago Chedraoui Silva  & RA: 082941\\
Raquel Mayumi Kawamoto & RA: 086003 \\    
  \end{tabular} \end{center}
    \itshape \today }\\[\baselineskip] \vspace{3.2pt} \endgroup}


\begin{document}
\titleTMB 
\newpage

%  \begin{abstract}
%=)
%  \end{abstract}
%\newpage

\tableofcontents

\newpage

\section{Introdução}
O propósito desse documento é descrever os requisitos do sistema
de gerenciamento de Conferências Tecnológicas
origanados pela técnica de levantamento de requisitos orientada a ponto de vista.
Dentre os principais assuntos do documento temos:

\begin{itemize}
\item Apresentar a técnica de levantamento de requisitos utilizada;
\item Mostrar a aplicação da técnica ao estudo de caso;
\item Descrição dos requisitos funcionais;
\item Descrição dos requisitos não-funcionais;
\item Descrições sobre as métricas para validação de requisitos.
\end{itemize}

\subsection{Escopo do produto }

O produto em questão é uma aplicação para gerenciamento de conferências tecnológicas que proporciona o suporte ao manejamento da alta quantidade de dados gerada pela conferência, tornando a manipulação desses dados mais fácil.
\subsubsection{Justificativa do Sistema}

Esse sistema se faz necessário uma vez que a quantidade de informação gerada em uma conferência tecnológica é muito grande, fazendo-se difícil a manipulação de tanto dado manualmente, assim como facilitar processos de submissão de papers, inscrições, pagamentos. O sistema se torna útil permitindo uma distribuição dos papers para os avaliadores de forma a evitar conflitos como um avaliador corrigir um paper de alguém conhecido, também se torna útil ao disponibilizar recursos com antecedência como, por exemplo, cronograma da conferência.


\subsubsection{Interação do usuário com o sistema}
O usuário acessa o sistema através da internet. Qualquer usuário podeconsultar o cronograma da conferência. O usuário pode ser dividido em 3 categorias (não mutuamente exclusivas): A primeira diz respeito a possibilidade de submissão de papers para avaliação mediante inscrição e pagamento da taxa de submissão. Usuários também podem se inscrever como participantes mediante, com a finalidade de participar de palestras, workshops e afins. Por último, um usuário pode somente se cadastrar no site para uso do fórum para, por exemplo, busca de informações específicas sobre o evento.

\subsubsection{Interação do sistema com outros sistemas}

O sistema interage com outros sistemas no momento em que um cadastro que possui uma taxa de admissão é realizado, o sistema pode entrar em contato com o sistema de cartão de crédito, gerenciamento de boleto, pag-seguro ou paypal, dependendo da forma que o usuário preferir realizar o pagamento.



\section{Descrição da técnica utilizada}

A técnica de levantamento de requisitos orientada a ponto de vista consiste em avaliar os diferentes interesses gerados pelos diferentes tipos de usuários finais. A análise dessa multi-perspectiva é importante, pois existe uma única maneira correta de analisar os requisitos do sistema. A técnica pode ser descrita em 4 etapas: Identificação, Estruturação, Documentação dos pontos de vistas e Mapeamento do Sistema conforme pontos de vistas.

A primeira consiste em descobrir os pontos de vista que utilizam quais serviços específicos. O segundo se baseia em agrupar pontos de vista relacionados, segundo uma hierarquia (serviços comuns localizados no nível mais alto e herdados por pontos de vista de nível inferior). Já o terceiro consiste em refinar a descrição dos pontos de vista e serviços identificados. Por último temos a quarta etapa que é utilizada para identificar objetos, utilizando informações de serviço encapsuladas nos pontos de vista.





\section{Aplicação da técnica ao estudo de caso}
\subsection{Resultados obtidos}
\subsection{Validação dos resultados}
\section{Requisitos}
\section{Requisitos Funcionais}
\subsection{Cadastro do Usuário}
\section{Requisitos não funcionais}
\subsection{Requisitos de produtos}
\subsection{Requisitos organizacionais}
\subsection{Requisitos externos}
\section{Métricas para especificar requisitos}
\section{Modelos do sistema}
\section{Evolução de sistema}
\section{Glossário}
\section{Referências}

\end{document}
