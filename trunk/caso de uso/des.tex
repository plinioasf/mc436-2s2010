\documentclass[11pt,a4paper,twoside]{article}
  \usepackage[T1]{fontenc}
  % \usepackage[applemac]{inputenc}
  \usepackage[latin1]{inputenc}

  %%%%%%%%%%%%%%%%%%%%%%%%%%%%%%%%%%%%%%%%%%%%%%%%%%%%%%%%%%%%%
  % Format de la  mise en page
  \setlength{\hoffset}{-1in}
  \setlength{\voffset}{-1in}
  \setlength{\oddsidemargin}{2cm}
  \setlength{\evensidemargin}{2cm}
  \setlength{\topmargin}{1.8cm}
  % \setlength{\headheight}{0cm}
  % \setlength{\headsep}{0cm}
  \setlength{\textwidth}{16.5cm}
  \setlength{\textheight}{24.0cm}


  \usepackage{pst-uml}
  
  
  % pour les environnement d'exemple Latex...
  \usepackage{fancyvrb}
  \usepackage[pstricks]{fvrb-ex}
  % option pour le package fancyverb (pour \VerbatimInput)
  \fvset{%
         frame=single,%
         numbers=left,%
		 baselinestretch=0.9,%
		 gobble=0,% nbr de caratere de debut a ignorer
		 fontsize=\footnotesize%
  }
  
  % \providecommand{\showgrid}{\psgrid[subgriddiv=0, griddots=10]}
  
  \DefineShortVerb{\|}
  % \UndefineShortVerb{\|}  % pour annuler 

  \pagestyle{headings}
  

  %%%%%%%%%%%%%%%%%%%%%%%%%%%%%%%%%%%%%%%%%%%%%%%%%%%%%%%%%%%%%%%%%%%%%% 
  % Quelques commandes locales
  %%%%%%%%%%%%%%%%%%%%%%%%%%%%%%%%%%%%%%%%%%%%%%%%%%%%%%%%%%%%%%%%%%%%%% 
  % \printtime
  % 
  % commande d'impression de l'heure courante
  % 
  % Exemple : "Fichier compil\'{e} le \today{} \`{a} \printtime."
  \usepackage{calc}
  \usepackage{ifthen}
  \newcounter{hours}\newcounter{minutes}
  \newcommand{\printtime}{%
    \setcounter{hours}{\time/60}%
    \setcounter{minutes}{\time-\value{hours}*60}%
    \thehours h%
    % on veut obtenir 15h03mn et non 15h3mn...
    \ifthenelse{\theminutes<10}{0}{}\theminutes mn%
  }
  
  %%%%%%%%%%%%%%%%%%%%%%%%%%%%%%%%%%%%%%%%%%%%%%%%%%%%%%%%%%%%%%%%%%%%%% 
  % Quelques abbreviations
  %%%%%%%%%%%%%%%%%%%%%%%%%%%%%%%%%%%%%%%%%%%%%%%%%%%%%%%%%%%%%%%%%%%%%% 
  
  % Conventions g\'{e}n\'{e}rales pour les formats de pr\'{e}sentation
  \newcommand{\strong}[1]{\textbf{\emph{#1}}} % plus fort que \emph
  \newcommand{\tech}[1]{\textsf{#1}}          % terme technique
  \newcommand{\file}[1]{\texttt{#1}} % noms de fichiers et de r\'{e}pertoires
  \newcommand{\menu}[1]{\fbox{#1}}       % nom d'un menu/sous-menu
  \newcommand{\key}[1]{\fbox{\textbf{#1}}}  % touche du clavier
  
  % abreviations locales a ce document :
  \newcommand{\uml}{\textsc{uml}} 
  \newcommand{\pstricks}{\texttt{PSTricks}} 
  \newcommand{\postscript}{\texttt{PostScript}} 
  \newcommand{\pstuml}{\texttt{pst-uml}} 
  % \newcommand{\bs}{\backslash}
  % \newcommand{\bs}{\backslash}
  
  % Pour les noms de commande TeX
  % Exemple : \cs{fbox} => \fbox
  \DeclareRobustCommand\cs[1]{\texttt{\char`\\#1}}

\usepackage[colorlinks,linktocpage]{hyperref}
%\usepackage{french}
\usepackage[francais]{babel} % idem frenchb mais PAS french !
  % La suite evite que Babel impose un espace devant ":" mais n'est
  % pas disponible sur les vielles versions de Babel (comme � l'ENSTA).
\NoAutoSpaceBeforeFDP
      
%%%%%%%%%%%%%%%%%%%%%%%%%%%%%%%%%%%%%%%%%%%%%%%%%%%%%%%%%%%%%%%%%%%%%%%%
%%%%%%%%%%%%%%%%%%%%%%%%%%%%%%%%%%%%%%%%%%%%%%%%%%%%%%%%%%%%%%%%%%%%%%%%
%%%%%%%%%%%%%%%%%%%%%%%%%%%%%%%%%%%%%%%%%%%%%%%%%%%%%%%%%%%%%%%%%%%%%%%%
\begin{document}

%%%%%%%%%%%%%%%%%%%%%%%%%%%%%%%%%%%%%%%%%%%%%%%%%%%%%%%%%%%%%%%%%%%%%%%%
%%%%%%%%%%%%%%%%%%%%%%%%%%%%%%%%%%%%%%%%%%%%%%%%%%%%%%%%%%%%%%%%%%%%%%%%
\title{Exemple de diagrammes utilisant  \pstuml}
\author{%
   Maurice \textsc{Diamantini}%
   \thanks{avec l'aide pr�cieuse de Denis \textsc{Girou}} %
   (email : \texttt{diam@ensta.fr})
}
\date{%
   Compil� le \today{} � \printtime{}.%
}
\maketitle

\tableofcontents
\clearpage
\section{Exemple de diagramme des cas d'utilisation}


%-%-%-%--%-%-%-%-%-%--%-%-%--%-%
% Author  : Tiago Chedraoui Silva
% License : GNU GPL v.3
% Title   : Mapa mental
% Tags    : mindmap, layers
%-%-%-%-%-%-%-%-%-%--%-%-%-%-%-%
\documentclass[11pt,a4paper,twoside]{article}
%\usepackage[paperheight=25cm,left=1cm,top=3cm]{geometry}
\usepackage[T1]{fontenc}
\usepackage[latin1]{inputenc}
\usepackage{pst-uml}
\begin{document}
\pagestyle{empty}

\begin{center}
  \resizebox{0.9\linewidth}{!}{%
    \begin{pspicture}(0,0.5)(15,13.5)%\psgrid
      \psset{framesep=0}
      \psframe[linewidth=0.5pt, linestyle=dashed](3,14)(12,0.5)
      \rput(7.5,1){\Large Sistema em desenvolvimento}
      \rput(1,12){\rnode{pl}{\umlActor{Participantes}}}
      \rput(1,6){\rnode{cl}{\umlActor{Coordenador}}}
      \rput(1,9){\rnode{al}{\umlActor{Avaliador}}}
      
      \rput(14,11.5){\rnode{ar}{\umlActor{Avaliador}}}
      \rput(14,6){\rnode{cr}{\umlActor{Coordenador}}}
      \rput(14,9){\rnode{pr}{\umlActor{Participantes}}}
      
      \rput(18,9){\rnode{usrl}{\umlActor{P�blico}}}
      \rput(-2,9){\rnode{usrr}{\umlActor{P�blico}}}
      
      \ncline{usrl}{ar}
      \ncputicon{umlHerit}
      \ncline{usrl}{pr}
      \ncputicon{umlHerit}
      \ncline{usrl}{cr}
      \ncputicon{umlHerit}

      \ncline{usrr}{pl}
      \ncputicon{umlHerit}
      \ncline{usrr}{cl}
      \ncputicon{umlHerit}
      \ncline{usrr}{al}
      \ncputicon{umlHerit}
      
      \umlPutCase{10,8}{lgi}{Login}
      \umlPutCase{10,10}{lgo}{Logout}
      \umlPutCase{5,3}{fdb}{Visualizar\\feedback}
      \umlPutCase{10,3}{vin}{Visualizar\\informa��es}
      \umlPutCase{5,13}{vce}{Vizualizar\\ certificados}
      \umlPutCase{5,7}{gce}{Gerar\\�certificados�}
      \umlPutCase{5,9}{vpa}{Validar\\papers�}
      \umlPutCase{5,11}{dfdb}{Dar\\ feedback�}
      \umlPutCase{8,7}{vts}{Vizualizar\\ trabalhos\\ submetidos}
      \ncline{pl}{dfdb}
      \ncline{cl}{gce}
      \ncline{al}{vpa}
      \ncline{pl}{vce}

      \ncSHS[armA=3.4]{usrl}{vin}
      \ncSHS[armA=3.4]{usrr}{fdb}
      \newpsstyle{umlDependance}{%
        linestyle=dashed,
        arrows=->,
        arrowscale=3,
        arrowinset=0.6
      }
      \ncline[style=umlDependance,offset=-0.5]{vts}{vpa}
      \naput{<<includes>>}
      
      % 
      % Login
      % 
      \ncline{ar}{lgi}
      \ncline{cr}{lgi}
      \ncline{pr}{lgi}

      % 
      % Logout
      % 
      \ncline{ar}{lgo}
      \ncline{cr}{lgo}
      \ncline{pr}{lgo}
    \end{pspicture}
  }%end resizeORscalebox
\end{center}


\end{document}

\end{document}


