%-%-%-%-%-%-%-%-%-%-%-%-%-%-%-%-%-%-%-%-%-%-%-%-%-%
% MC436 Engenharia de Software                    %  
% Projeto : Sistema de gerenciamento              %
% de conferências técnicas                        %
% Data:18/08/2010                                 %
% Unicamp,Campinas,São Paulo,Brasil               % 
% Grupo:                                          %
% - Mariana Mendes Caspirro                       % 
% - Murilo Fossa Vicentini                        % 
% - Plinio Augusto Soriano Freire                 %
% - Tiago Chedraoui Silva                         %
% - Raquel Mayume Kawamoto                        % 
%-%-%-%-%-%-%-%-%-%-%-%-%-%-%-%-%-%-%-%-%-%-%-%-%-%
\documentclass[letter]{article}

%%% fontes %%%
\usepackage[T1]{fontenc}
\usepackage[brazil]{babel}    % dá suporte para os termos na língua portuguesa do Brasi
\usepackage[utf8]{inputenc}   % acentuação
\usepackage{ae,aecompl,aeguill}       % pdfs mais bonitos =)

%%% outros %%%
\usepackage{multirow}
\usepackage{textcomp}
\usepackage{color}       
\usepackage{indentfirst}      % retira padrao americano de paragrafos
\usepackage{multicol}   
\usepackage[linkbordercolor={1 1 1},urlcolor=black,colorlinks=false]{hyperref} % links


\usepackage{cite}  % Needed to use citations.  
\usepackage{indentfirst} %retira padrao americano de paragrafos
\usepackage{ae,aecompl}
\usepackage{pslatex}
\usepackage{epsfig}
\usepackage{verbatim}
\usepackage{pstricks}
\usepackage[letterpaper]{geometry}
\geometry{verbose,lmargin=3cm,rmargin=3cm}


%%% extras %%%
\RequirePackage{marvosym} % figuras \Letter \Email 
\usepackage{fancyhdr}     % Headers
\usepackage{epsf}


% Capa estilizada %
\newcommand*{\titleTMB}{\begingroup \centering \settowidth{\unitlength}{\LARGE MC 613} \vspace*{\baselineskip} {\large\scshape  Turma A}\\[\baselineskip] \rule{11.0cm}{1.6pt}\vspace*{-\baselineskip}\vspace*{2pt} \rule{11.0cm}{0.4pt}\\[\baselineskip] {\LARGE Levantamento de requisitos para }\\[0.2\baselineskip] {\LARGE Sistema de gerenciamento de Conferências Tecnológicas }\\[0.2\baselineskip] {\itshape MC 436 - Engenharia de Software - Segundo Semestre de 2010}\\[0.2\baselineskip] \rule{11.0cm}{0.4pt}\vspace*{-\baselineskip}\vspace{3.2pt} \rule{11.0cm}{1.6pt}\\[\baselineskip] {\large\scshape Professora: Ariadne Maria Brito Rizzoni Carvalho}\par \vfill {\normalsize   \scshape 
    \begin{center} 
      \begin{tabular}{  l  l  p{5cm} } 
        Mariana Mendes Caspirro & RA: 082204\\
        Murilo Fossa Vicentini & RA: 082335 \\
        Plinio Augusto Soriano Freire & RA:  082505\\
        Tiago Chedraoui Silva  & RA: 082941\\
Raquel Mayumi Kawamoto & RA: 086003 \\    
  \end{tabular} \end{center}
    \itshape \today }\\[\baselineskip] \vspace{3.2pt} \endgroup}




\begin{document}
\titleTMB 
\newpage


\tableofcontents

\newpage

\section{Introdução}
O propósito desse documento é descrever os requisitos do sistema
de gerenciamento de Conferências Tecnológicas
originados pela técnica de levantamento de requisitos orientada a ponto de vista.
Dentre os principais assuntos do documento temos:

\begin{itemize}
\item Apresentar a técnica de levantamento de requisitos utilizada;
\item Mostrar a aplicação da técnica ao estudo de caso;
\item Descrição dos requisitos funcionais;
\item Descrição dos requisitos não-funcionais;
\item Descrições sobre as métricas para validação de requisitos.
\end{itemize}

\subsection{Escopo do produto }

O produto em questão é uma aplicação para gerenciamento de conferências tecnológicas que proporciona o suporte ao manejamento da alta quantidade de dados gerada pela conferência, tornando a manipulação desses dados mais fácil.
\subsubsection{Justificativa do Sistema}

Esse sistema se faz necessário uma vez que a quantidade de informação gerada em uma conferência tecnológica é muito grande, fazendo-se difícil a manipulação de tanto dado manualmente, assim como facilitar processos de submissão de papers, inscrições, pagamentos. O sistema se torna útil permitindo uma distribuição dos papers para os avaliadores de forma a evitar conflitos como um avaliador corrigir um paper de alguém conhecido, também se torna útil ao disponibilizar recursos com antecedência como, por exemplo, cronograma da conferência.


\subsubsection{Interação do usuário com o sistema}
O usuário acessa o sistema através da internet. Qualquer usuário pode consultar o cronograma da conferência. O usuário pode ser dividido em 3 categorias (não mutuamente exclusivas): A primeira diz respeito a possibilidade de submissão de papers para avaliação mediante inscrição e pagamento da taxa de submissão. Usuários também podem se inscrever como participantes mediante, com a finalidade de participar de palestras, workshops e afins. Por último, um usuário pode somente se cadastrar no site para uso do fórum para, por exemplo, busca de informações específicas sobre o evento.

\subsubsection{Interação do sistema com outros sistemas}

O sistema interage com outros sistemas no momento em que um cadastro que possui uma taxa de admissão é realizado, o sistema pode entrar em contato com o sistema de cartão de crédito, gerenciamento de boleto, pag-seguro ou paypal, dependendo da forma que o usuário preferir realizar o pagamento.



\section{Descrição da técnica utilizada}

A técnica de levantamento de requisitos orientada a ponto de vista consiste em avaliar os diferentes interesses gerados pelos diferentes tipos de usuários finais. A análise dessa multi-perspectiva é importante, pois existe uma única maneira correta de analisar os requisitos do sistema. A técnica pode ser descrita em 4 etapas: Identificação, Estruturação, Documentação dos pontos de vistas e Mapeamento do Sistema conforme pontos de vistas.

A primeira consiste em descobrir os pontos de vista que utilizam quais serviços específicos. O segundo se baseia em agrupar pontos de vista relacionados, segundo uma hierarquia (serviços comuns localizados no nível mais alto e herdados por pontos de vista de nível inferior). Já o terceiro consiste em refinar a descrição dos pontos de vista e serviços identificados. Por último temos a quarta etapa que é utilizada para identificar objetos, utilizando informações de serviço encapsuladas nos pontos de vista.





\section{Mapeamento Pontos de Vistas/Serviços}

Com o intuito de reconhecer as várias perspectivas e descobrir conflitos nos requisitos propostos, é realizado uma abordagem orientada a ponto de vista do sistema gerenciador de conferências tecnológicas.


\begin{center}

\begin{table}[h!]

\begin{center}
\begin{tabular}{|ll|}
\hline 
\multicolumn{2}{|c|}{\textbf{Participantes}}\tabularnewline
\hline
\textbf{Referência:} & Participantes\tabularnewline
\textbf{Atributos} & Número de inscrição\tabularnewline
\textbf{Eventos} & \tabularnewline
\textbf{Serviços} & Inscrição\tabularnewline
 & Submissão\tabularnewline
 & Avaliar e enviar comentários após apresentação\tabularnewline
\textbf{Subpontos de vista:} & Apresentadores \tabularnewline
\hline
\end{tabular}
\end{center}
\end{table}


%
\begin{table}[h!]


\begin{center}
\begin{tabular}{|ll|}
\hline 
\multicolumn{2}{|c|}{\textbf{Apresentadores}}\tabularnewline
\hline
\textbf{Referência:} & Apresentadores\tabularnewline
\textbf{Atributos} & Número de palestras/mini-cursos\tabularnewline
\textbf{Eventos} & \tabularnewline
\textbf{Serviços} & Editar trabalho submetido\tabularnewline
\textbf{Subpontos de vista:} & Apresentadores \tabularnewline
\hline
\end{tabular}

\end{center}
\end{table}


%
\begin{table}[h!]
\begin{center}
\begin{tabular}{|ll|}
\hline 
\multicolumn{2}{|c|}{\textbf{Avaliadores}}\tabularnewline
\hline
\textbf{Referência:} & Avaliadores\tabularnewline
\textbf{Atributos} & id\_avaliador\tabularnewline
\textbf{Eventos} & \tabularnewline
\textbf{Serviços} & Avaliar trabalhos\tabularnewline
 & Recusar avaliação\tabularnewline
\textbf{Subpontos de vista:} & Coordenadores\tabularnewline
\hline
\end{tabular}
\end{center}
\end{table}
%
\newpage
\begin{table}[h!]
\begin{center}
\begin{tabular}{|ll|}
\hline 
\multicolumn{2}{|c|}{\textbf{Organizadores}}\tabularnewline
\hline
\textbf{Referência:} & Organizadores\tabularnewline
\textbf{Atributos} & Função, id\_organizador\tabularnewline
\textbf{Eventos} & \tabularnewline
\textbf{Serviços} & Consultar estatísticas sobre gastos\tabularnewline
\textbf{Subpontos de vista:} & Coordenadores\tabularnewline
\hline
\end{tabular}
\end{center}
\end{table}


%
\begin{table}[h!]
\begin{center}
\begin{tabular}{|ll|}
\hline 
\multicolumn{2}{|c|}{\textbf{Coordenadores}}\tabularnewline
\hline
\textbf{Referência:} & Coordenadores\tabularnewline
\textbf{Atributos} & Número de palestras/mini-cursos\tabularnewline
\textbf{Eventos} & \tabularnewline
\textbf{Serviços} & Alterar calendário\tabularnewline
 & Classificar palestras\tabularnewline
 & Transformar documentos dinâmicos em estáticos\tabularnewline
 & Divulgar slides e notas\tabularnewline
\textbf{Subpontos de vista:} & Apresentadores \tabularnewline
\hline
\end{tabular}
\end{center}
\end{table}


%
\begin{table}[h!]
\begin{center}
\begin{tabular}{|ll|}
\hline 
\multicolumn{2}{|c|}{\textbf{Público}}\tabularnewline
\hline
\textbf{Referência:} &  Público\tabularnewline
\textbf{Atributos} & \tabularnewline
\textbf{Eventos} & \tabularnewline
\textbf{Serviços} & \tabularnewline
 & Consultar informações de local/transporte\tabularnewline
 & Consultar estatísticas de fotos\tabularnewline
 & Acompanhar calendário \tabularnewline
 & Consultar mudanças\tabularnewline
 & Consultar palestrantes convidados\tabularnewline
 & Consultar palestras\tabularnewline
 & Consultar publicações\tabularnewline
 & Acessar galeria\tabularnewline
\textbf{Subpontos de vista:} & Participantes, Organizadores, Avaliadores \tabularnewline
\hline

\end{tabular}

\end{center}

\end{table}

\end{center}

%%%%%%%%%%%%%%%%%%%%%%%Mapeamento 

\newpage
\section{Requisitos Funcionais}

Os requisitos funcionais do sistema é um dos resultados da aplicação de uma técnica de levantamento de requisitos. Esses requisitos funcionais descrevem as funções que o sistema deve fornecer e como lidar com as entradas e saídas, assim como apresentam os templates associados a cada ponto de vista/serviço. Os serviços disponíveis estão dispostos a seguir:

\subsection{Inscrição Usuário}



%
\begin{table}[h!]\begin{center}
\begin{tabular}{|ll|}
\hline 
\multicolumn{2}{|c|}{\textbf{Inscrição}}\tabularnewline
\hline
\textbf{Referência:} & Inscrever participante\tabularnewline
\textbf{Razão:} & Facilidade para os participantes\tabularnewline
\textbf{Especificações:} & Entrar na área de cadastro\tabularnewline
 & Digitar informações e efetuar o pagamento\tabularnewline
\textbf{Pontos de vista:} & Participantes\tabularnewline
\textbf{Requisito não funcional:} & Garantir privacidade das informações pessoais dos participantes \tabularnewline
 & Garantir a segurança ao efetuar o pagamento\tabularnewline
\textbf{Provedor:} & \tabularnewline
\hline\end{tabular}\end{center}
\end{table}


\textbf{Função:} Inscrição Usuário.

\textbf{Fase: } Pré-Conferência.

\textbf{Descrição: } O sistema deve permitir que o usuário se cadastre no site via pagamento da taxa de admissão.

\textbf{Entrada: } Dados do Usuário.

\textbf{Saída: :} Confirmação de cadastro e identificador único gerado

\textbf{Requer:} Comunicação com a operadora de cartões, caso seja essa a
forma de pagamento escolhida.



\subsection{ Submissão de Paper}


%
\begin{table}[h!]\begin{center}

\begin{center}
\begin{tabular}{|ll|}
\hline 
\multicolumn{2}{|c|}{\textbf{Submissão}}\tabularnewline
\hline
\textbf{Referência:} & Submeter trabalhos\tabularnewline
\textbf{Razão:} & Facilidade para os participantes\tabularnewline
\textbf{Especificações:} & Realizar upload de trabalhos na área apropriada\tabularnewline
\textbf{Pontos de vista:} & Participantes\tabularnewline
\textbf{Requisito não funcional:} & Garantir segurança. Isto é, apenas as pessoas autorizadas devem ter
acesso aos trabalhos\tabularnewline
\textbf{Provedor:} & \tabularnewline
\hline\end{tabular}\end{center}

\end{center}
\end{table}


\textbf{Função:} Submissão

\textbf{Fase: } Pré-Conferência

\textbf{Descrição: } O sistema deve permitir que o participante submeta papers via pagamento da taxa de submissão

\textbf{Pré-condição: } O usuário deverá estar identificado no sistema com seu login

\textbf{Entrada: } Identificador do Participante, Paper

\textbf{Saída: :} Confirmação da submissão

\textbf{Requer:} Comunicação com a operadora de cartões, caso seja essa a forma de pagamento escolhida.

\subsection{ Avaliação e Envios de Comentários}

%
\begin{table}[h!]
\begin{center}
\begin{tabular}{|ll|}
\hline 
\multicolumn{2}{|c|}{\textbf{Comentar apresentação}}\tabularnewline
\hline
\textbf{Referência:} & Comentar apresentação\tabularnewline
\textbf{Razão:} & Feedback para os apresentadores\tabularnewline
\textbf{Especificações:} & Entrar na página de uma determinada apresentação\tabularnewline
& e deixar comentários
ou dúvidas para o autor\tabularnewline
\textbf{Pontos de vista:} & Participantes\tabularnewline
\textbf{Requisito não funcional:} & \tabularnewline
\textbf{Provedor:} & \tabularnewline
\hline
\end{tabular}
\end{center}
\end{table}


\textbf{Função:} Comentários

\textbf{Fase: } Conferência

\textbf{Descrição: } O sistema deve permitir que o participante possa avaliar e enviar comentários para outros participantes ou apresentadores.

\textbf{Pré-condição: } O usuário deverá estar identificado no sistema com seu login

\textbf{Entrada: } Avaliação/Comentário gerado pelo participante

\textbf{Saída: :} Disponibilização da entrada para os destinatários escolhidos pelo participante


\subsection{ Consulta a informações de local/transporte}

\begin{table}[h!]\begin{center}
\begin{tabular}{|ll|}
\hline 
\multicolumn{2}{|c|}{\textbf{Consultar informações}}\tabularnewline
\hline
\textbf{Referência:} & Consultar local/transporte\tabularnewline
\textbf{Razão:} & Informação para o público\tabularnewline
\textbf{Especificações:} & Entrar na seção de informações sobre o local/transporte\tabularnewline
 & observar mapas\tabularnewline
 & informações sobre transporte público\tabularnewline
\textbf{Pontos de vista:} & Público\tabularnewline
\textbf{Requisito não funcional:} & \tabularnewline
\textbf{Provedor:} & \tabularnewline
\hline\end{tabular}\end{center}
\end{table}

\textbf{Função:} Consulta a informações

\textbf{Fase: } Pré-Conferência, Conferência

\textbf{Descrição: } O sistema deve disponibilizar no site principais informações sobre meios de transporte, hotéis, mapas e afins.

\textbf{Entrada: } Nenhuma entrada é requerida, sendo somente necessário a abertura do site.

\textbf{Saída: :} Uma lista de hotéis e transportes para a conferência, assim como mapas para o local do evento.

\newpage
\subsection{ Busca de Eventos}

\begin{table}[h!]\begin{center}
\begin{tabular}{|ll|}
\hline 
\multicolumn{2}{|c|}{\textbf{Consultar palestras}}\tabularnewline
\hline
\textbf{Referência:} & Consultar palestras\tabularnewline
\textbf{Razão:} & Informação para o público\tabularnewline
\textbf{Especificações:} & Entrar na seção de informações sobre palestras\tabularnewline
 & visualizar as palestrar por áreas de interesse\tabularnewline
\textbf{Pontos de vista:} & Público\tabularnewline
\textbf{Requisito não funcional:} & \tabularnewline
\textbf{Provedor:} & \tabularnewline
\hline\end{tabular}\end{center}
\end{table}



\textbf{Função:} Busca de Eventos

\textbf{Fase: } Pré-Conferência, Conferência

\textbf{Descrição: } O sistema deve possuir um sistema de busca para que palestras, workshops e afins sejam achados pelo usuário. Nesse sistema de busca o usuário escolhe alguns atributos de pesquisa conforme sua preferência para achar o evento.

\textbf{Entrada: } Conjunto de preferências do usuário em relação ao evento.

\textbf{Saída: :} O sistema disponibiliza para o usuário uma lista de eventos conforme às preferências apontadas.



\subsection{ Exibir Calendário}


\begin{flushleft}
%
\begin{table}[h!]\begin{center}
\begin{tabular}{|ll|}
\hline 
\multicolumn{2}{|c|}{\textbf{Acompanhar calendário}}\tabularnewline
\hline
\textbf{Referência:} & Acompanhar calendário\tabularnewline
\textbf{Razão:} & Informação para o público\tabularnewline
\textbf{Especificações:} & Acesso direto pelo website\tabularnewline
 & Possibilidade de visualização de informações separadas por dia (ou
como um todo)\tabularnewline
\textbf{Pontos de vista:} & Público\tabularnewline
\textbf{Requisito não funcional:} & \tabularnewline
\textbf{Provedor:} & \tabularnewline
\hline\end{tabular}\end{center}
\end{table}

\par\end{flushleft}


%



\textbf{Função:} Exibição de eventos

\textbf{Fase: } Pré-Conferência, Conferência

\textbf{Descrição: } O sistema deve disponibilizar no site principal informações sobre alguns dos eventos disponíveis ou recomendados, contendo seu 
nome, data e categoria na qual está incluso.

\textbf{Entrada: } Nenhuma entrada é requerida, sendo somente necessário a
abertura do site.

\textbf{Saída: :} Uma lista dos eventos.


\newpage
\subsection{Informações sobre publicações}

%
\begin{table}[h!]\begin{center}
\begin{tabular}{|ll|}
\hline 
\multicolumn{2}{|c|}{\textbf{Consultar publicações}}\tabularnewline
\hline
\textbf{Referência:} & Consultar publicações\tabularnewline
\textbf{Razão:} & Informação para o público\tabularnewline
\textbf{Especificações:} & Acesso direto pelo website\tabularnewline
 & Links para baixar ou comprar publicações referentes à conferência\tabularnewline
\textbf{Pontos de vista:} & Público\tabularnewline
\textbf{Requisito não funcional:} & Deve ter destaque na página de modo que seja de fácil visualização\tabularnewline
\textbf{Provedor:} & \tabularnewline
\hline\end{tabular}\end{center}
\end{table}



\textbf{Função:} Exibição de informações sobre publicações

\textbf{Fase: } Pós-Conferência

\textbf{Descrição: } O sistema deve disponibilizar no site principal as informações sobre as publicações participantes do evento, contendo o nome do autor, título, categoria e breve descrição.

\textbf{Entrada: } Nenhuma entrada é requerida, sendo somente necessário a
abertura do site.

\textbf{Saída: :} Uma lista de características de publicações.



\subsection{Editar Trabalhos Submetidos}

%
\begin{table}[h!]
\begin{center}
\begin{tabular}{|ll|}
\hline 
\multicolumn{2}{|c|}{\textbf{Editar}}\tabularnewline
\hline
\textbf{Referência:} & Editar o trabalho\tabularnewline
\textbf{Razão:} & Assistência aos apresentadores\tabularnewline
\textbf{Especificações:} & Realizar upload do trabalho editado na área apropriada\tabularnewline
\textbf{Pontos de vista:} & Apresentador\tabularnewline
\textbf{Requisito não funcional:} & Garantir segurança no acesso\tabularnewline
\textbf{Provedor:} & \tabularnewline
\hline
\end{tabular}
\end{center}
\end{table}


\textbf{Função:}  Editar

\textbf{Fase:} Pré-Conferência

\textbf{Descrição:} O sistema deve permitir que o usuário possa editar seu trabalho submetido, afim de gerar melhorias, mudar padrões, entre outros.

\textbf{Pré-condição:}  O usuário deverá estar identificado no sistema com seu login e possuir um trabalho previamente submetido.

\textbf{Entrada:}  Nenhuma entrada é requerida.

\textbf{Saída:}  Trabalho atualizado.


\newpage
\subsection{Acesso a Galeria de Fotos/Vídeos}

%
\begin{table}[h!]\begin{center}
\begin{tabular}{|ll|}
\hline 
\multicolumn{2}{|c|}{\textbf{Acessar galeria de fotos}}\tabularnewline
\hline
\textbf{Referência:} & Acessar galeria de fotos\tabularnewline
\textbf{Razão:} & Informação para o público\tabularnewline
\textbf{Especificações:} & Acesso direto pelo website\tabularnewline
 & Visualização da galeria de fotos com a possibilidade de classificação
por data\tabularnewline
\textbf{Pontos de vista:} & Público\tabularnewline
\textbf{Requisito não funcional:} & \tabularnewline
\textbf{Provedor:} & \tabularnewline
\hline\end{tabular}\end{center}
\end{table}



\textbf{Função:} Acesso a Galeria

\textbf{Fase: } Pós-Conferência

\textbf{Descrição: } O sistema deve disponibilizar no site principal fotos e vídeos gerados durante a realização

\textbf{Entrada: } Nenhuma entrada é requerida, sendo somente necessário a abertura do site.

\textbf{Saída: :} Fotos e Videos da conferência.



\subsection{ Avaliação de Trabalhos}

%
\begin{table}[h!]\begin{center}
\begin{tabular}{|ll|}
\hline 
\multicolumn{2}{|c|}{\textbf{Avaliar trabalhos}}\tabularnewline
\hline
\textbf{Referência:} & Avaliar trabalhos\tabularnewline
\textbf{Razão:} & Facilidade para os avaliadores\tabularnewline
\textbf{Especificações:} & Entrar com login e senha na seção de acesso dos avaliadores\tabularnewline
 & Escolher uma área de interesse (assunto do trabalho)\tabularnewline

 & Avaliar trabalho\tabularnewline
\textbf{Pontos de vista:} & Avaliadores\tabularnewline
\textbf{Requisito não funcional:} & Desígnio de trabalho para avaliadores deve ser de forma aleatória\tabularnewline
 & Sorteio após no máximo um minuto da escolha da área de interesse\tabularnewline
\textbf{Provedor:} & \tabularnewline
\hline\end{tabular}\end{center}
\end{table}



\textbf{Função:} Avaliação de Trabalhos

\textbf{Fase: } Pré-Conferência

\textbf{Descrição: } O sistema deve permitir a avaliação de trabalhos submetidos por parte dos avaliadores

\textbf{Entrada: } Trabalho avaliado pelo avaliador com os devidos comentários

\textbf{Saída: :} Disponibilização da entrada para o autor do paper avaliado e para o coordenador.

\newpage
\subsection{ Recusa de Avaliação}

\begin{table}[h!]\begin{center}
\begin{tabular}{|ll|}
\hline 
\multicolumn{2}{|c|}{\textbf{Recusar avaliação}}\tabularnewline
\hline
\textbf{Referência:} & Recusar avaliação\tabularnewline
\textbf{Razão:} & Facilidade para os avaliadores\tabularnewline
\textbf{Especificações:} & Escolher opção \textquotedblleft{}rejeitar\textquotedblright{} para
outro trabalho ser sorteado\tabularnewline
\textbf{Pontos de vista:} & Avaliador\tabularnewline
\textbf{Requisito não funcional:} & Novo trabalho deve ser sorteado após no máximo um minuto da desistência
anterior\tabularnewline
\textbf{Provedor:} & \tabularnewline
\hline\end{tabular}\end{center}
\end{table}


\textbf{Função:} Recusa de Avaliação

\textbf{Fase: } Pré-Conferência

\textbf{Descrição: } O sistema deve permitir que um avaliador recuse a avaliação de um trabalho submetido mediante justificativa

\textbf{Entrada: } Justificativa para a recusa de avaliação do trabalho

\textbf{Saída: :} Redirecionamento do trabalho para outro avaliador




\subsection{Consultar Gastos}

%
\begin{table}[h!]\begin{center}
\begin{tabular}{|ll|}
\hline 
\multicolumn{2}{|c|}{\textbf{Consultar gastos}}\tabularnewline
\hline
\textbf{Referência:} & Consultar gastos\tabularnewline
\textbf{Razão:} & Informação para os organizadores\tabularnewline
\textbf{Especificações:} & Acesso somente aos organizadores\tabularnewline
 & Possibilidade de visualização das estatísticas sobre gastos\tabularnewline
\textbf{Pontos de vista:} & Organizadores\tabularnewline
\textbf{Requisito não funcional:} & \tabularnewline
\textbf{Provedor:} & \tabularnewline
\hline\end{tabular}\end{center}
\end{table}


\textbf{Função:}  Consultar Gastos

\textbf{Fase:}  Pós-Conferência

\textbf{Descrição: } O sistema deve permitir que os organizadores do evento possam ter um acesso ao balanço final de gastos da conferência.

\textbf{Pré-condição:}  O organizador deverá estar identificado no sistema com seu login.

\textbf{Entrada: } Nenhuma entrada é requerida.

\textbf{Saída:}  Tabela com os dados dos gastos gerados pela conferência.

\newpage
\subsection{Consultar Mudanças}

%
\begin{table}[h!]\begin{center}
\begin{tabular}{|ll|}
\hline 
\multicolumn{2}{|c|}{\textbf{Consultar mudanças}}\tabularnewline
\hline
\textbf{Referência:} & Consultar mudanças\tabularnewline
\textbf{Razão:} & Informação para o público\tabularnewline
\textbf{Especificações:} & Acesso direto pelo website\tabularnewline
 & Observação do quadro de avisos\tabularnewline
\textbf{Pontos de vista:} & Público\tabularnewline
\textbf{Requisito não funcional:} & Deve ter destaque na página de modo que seja de fácil visualização\tabularnewline
\textbf{Provedor:} & \tabularnewline
\hline\end{tabular}\end{center}
\end{table}


\textbf{Função:}  Consultar Mudanças

\textbf{Fase:}  Conferência

\textbf{Descrição:}  O sistema deve permitir que o público possa visualizar as mudanças de cronograma geradas durante a conferência que estarão localizadas em destaque na página

\textbf{Entrada:}  Nenhuma entrada é requerida, sendo somente necessário a abertura do site.

\textbf{Saída:}  Lista com as mudanças geradas durante a conferência.

\subsection{Consultar Palestrantes Convidados}

%
\begin{table}[h!]\begin{center}
\begin{tabular}{|ll|}
\hline 
\multicolumn{2}{|c|}{\textbf{Consultar palestrantes convidados}}\tabularnewline
\hline
\textbf{Referência:} & Consultar palestrantes convidados\tabularnewline
\textbf{Razão:} & Informação para o público\tabularnewline
\textbf{Especificações:} & Entrar na seção de informações sobre palestrantes convidados\tabularnewline
 & observar perfis\tabularnewline
 & páginas da web e outras informações disponíveis sobre os palestrantes\tabularnewline
\textbf{Pontos de vista:} & Público\tabularnewline
\textbf{Requisito não funcional:} & \tabularnewline
\textbf{Provedor:} & \tabularnewline
\hline\end{tabular}\end{center}
\end{table}


\textbf{Função:}  Consultar Palestrantes Convidados

\textbf{Fase:}  Pré-Conferência

\textbf{Descrição:}  O sistema deverá permitir que o público busque pelos palestrantes onde haverá informações sobre página web do palestrante, perfil e outras informações.

\textbf{Entrada:}  Conjunto de Informações para realização da busca

\textbf{Saída:}  O sistema disponibiliza para o usuário uma lista conforme às informações apontadas.

\newpage
\subsection{Consultar estatísticas}


\begin{table}[h!]\begin{center}
\begin{tabular}{|ll|}
\hline 
\multicolumn{2}{|c|}{\textbf{Consultar estatísticas}}\tabularnewline
\hline
\textbf{Referência:} & Consultar estatísticas\tabularnewline
\textbf{Razão:} & Informação para o público\tabularnewline
\textbf{Especificações:} & Acesso direto pelo website\tabularnewline
 & Possibilidade de visualização das estatísticas (número de participantes,
público diário,\tabularnewline
 & avaliação das apresentações, popularidade de tópicos)\tabularnewline
\textbf{Pontos de vista:} & Público\tabularnewline
\textbf{Requisito não funcional:} & \tabularnewline
\textbf{Provedor:} & \tabularnewline
\hline\end{tabular}\end{center}
\end{table}


\textbf{Função:}  Consultar estatísticas

\textbf{Fase:}  Pós- Conferência

\textbf{Descrição:}  O sistema deve disponibilizar no site principal informações sobre o número de pessoas que participaram do evento, da quantidade de palestrantes, dos usuários que visitaram o site durante a conferência, entre outros.

\textbf{Entrada:}  Nenhuma entrada é requerida, sendo somente necessário a abertura do site.

\textbf{Saída:}  Tabela com as estatísticas.
 

\subsection{Alterar Calendário}


%
\begin{table}[h!]

\begin{center}
\begin{tabular}{|ll|}
\hline 
\multicolumn{2}{|c|}{\textbf{Alterar calendário}}\tabularnewline
\hline
\textbf{Referência:} & Alterar calendário\tabularnewline
\textbf{Razão:} & Facilidade para os coordenadores, assim como para os participantes\tabularnewline
\textbf{Especificações:} & Com acesso restrito aos coordenadores (login e senha), é possível
alterar o calendário.\tabularnewline
 &  A alteração deve também ser apresentada num quadro de avisos com
destaque no website.\tabularnewline
\textbf{Pontos de vista:} & Coordenadores \tabularnewline
\textbf{Requisito não funcional:} & As mudanças devem ser atualizadas de imediato\tabularnewline
\textbf{Provedor:} & \tabularnewline
\hline
\end{tabular}
\end{center}
\end{table}


\textbf{Função:} Alterar Calendário

\textbf{Fase:}  Conferência

\textbf{Descrição:}  O sistema deve permitir que o coordenador possa modificar o calendário devido a imprevistos ocorridos durante a conferência

\textbf{Pré-condição:}  O coordenador deverá estar identificado no sistema com seu login.

\textbf{Entrada:} Dados a serem alterados no calendário.

\textbf{Saída:}  Mudanças em destaque no site e o calendário atualizado.


\newpage
\subsection{Classificar palestras}



\begin{table}[h!]

\begin{center}
\begin{tabular}{|ll|}
\hline 
\multicolumn{2}{|c|}{\textbf{Classificar palestras}}\tabularnewline
\hline
\textbf{Referência:} & Classificar palestras por áreas de interesse\tabularnewline
\textbf{Razão:} & Facilidade para os coordenadores, assim como para os participantes\tabularnewline
\textbf{Especificações:} & Coordenadores classificam palestras em relação às áreas as quais\tabularnewline
 &  estas pertencem (utilização de palavras-chave)\tabularnewline
\textbf{Pontos de vista:} & Coordenadores\tabularnewline
\textbf{Requisito não funcional:} & \tabularnewline
\textbf{Provedor:} & \tabularnewline
\hline
\end{tabular}
\end{center}
\end{table}

\textbf{Função:} Classificar palestras quanto às suas áreas/categorias

\textbf{Fase:}  Pré-Conferência

\textbf{Descrição:}  O sistema deve permitir que o coordenador possa ordenar as palestras conforme sua categoria/área.

\textbf{Pré-condição:}  O coordenador deverá estar identificado no sistema com seu login.

\textbf{Entrada:}  Palestras a serem classificadas

\textbf{Saída:}  Tabelas das palestras classificadas de acordo com sua categoria



\subsection{Transformar Documentos Dinâmicos em Estáticos}



%
\begin{table}[h!]
\begin{center}
\begin{tabular}{|ll|}
\hline 
\multicolumn{2}{|c|}{\textbf{Transformar documentos dinâmicos em estáticos}}\tabularnewline
\hline
\textbf{Referência:} & Transformar documentos dinâmicos em estáticos\tabularnewline
\textbf{Razão:} & Encerramento da conferência\tabularnewline
\textbf{Especificações:} & Coordenadores impedem que sejam editados\tabularnewline
& ou inseridos novos conteúdos (trabalhos, notas, etc.)\tabularnewline
 &  no website após uma determinada data\tabularnewline
\textbf{Pontos de vista:} & Coordenadores\tabularnewline
\textbf{Requisito não funcional:} & \tabularnewline
\textbf{Provedor:} & \tabularnewline
\hline
\end{tabular}
\end{center}
\end{table}

\textbf{Função:} Transformar Documentos Dinâmicos em Estáticos

\textbf{Fase:}  Pós-Conferência

\textbf{Descrição:}  O sistema deve permitir que o coordenador transforme os dados dinâmicos em estáticos, ou seja, dados que são constantemente gerados durante a conferência como, por exemplo, estatísticos.

\textbf{Pré-condição:}  O coordenador deverá estar identificado no sistema com seu login.

\textbf{Entrada:}  Documentos Dinâmicos

\textbf{Saída:}  Documentos Estáticos

\newpage
\subsection{Divulgar slides e notas}


\begin{table}[h!]

\begin{center}
\begin{tabular}{|ll|}
\hline 
\multicolumn{2}{|c|}{\textbf{Divulgar slides e notas}}\tabularnewline
\hline
\textbf{Referência:} & Divulgar slides e notas\tabularnewline
\textbf{Razão:} & Encerramento da conferência\tabularnewline
\textbf{Especificações:} &  Coordenadores disponibilizam no website o conteúdo
(estático) livremente\tabularnewline
 & ou inserem informações de como adquirir publicações referentes a conferência\tabularnewline
\textbf{Pontos de vista:} & Coordenadores\tabularnewline
\textbf{Requisito não funcional:} & \tabularnewline
\textbf{Provedor:} & \tabularnewline
\hline
\end{tabular}

\end{center}
\end{table}


\textbf{Função:} Divulgar slides e notas

\textbf{Fase:}  Conferência

\textbf{Descrição:}  O sistema deve permitir que o coordenador possa divulgar, a partir do consentimento do palestrante, slides utilizados durante a conferência e informações adicionais.

\textbf{Pré-condição:}  O coordenador deverá estar identificado no sistema com seu login e o palestrante deve autorizar a disponibilização dos materiais.

\textbf{Entrada:}  Dados e documentos a serem inseridos.

\textbf{Saída:}  Mudanças devem aparecer no site na seção de cada palestrante.

%

 





\section{Requisitos não funcionais}

\subsection{ Facilidade de uso}
\textbullet Tempo de treinamento

O tempo máximo para que um usuário aprenda a utilizar o sistema não deve ser superior a 15 minutos.
Supõe-se que em tal tempo um usuário seja capaz de entrar ao site 
pela primeira vez e realizar a inscrição e buscar informações necessárias para a sua participação .


\subsection{Confiabilidade}
\textbullet\hspace{1mm} Taxa de ocorrência de falhas

A quantidade de falhas apresentadas pelo sistema deve ser inferior a 5 falhas\ mês.

\textbullet\hspace{1mm} Disponibilidade

O sistema deve estar disponível todo o tempo, com a tolerância mínima para as falhas.


\subsection{Robustez}
\textbullet\hspace{1mm} Tempo de reinício

Após uma falha, durante o período de pré-conferência, o sistema deverá voltar a estar disponível em um tempo t, tal que t $\leq$ 15 minutos .
Após uma falha, durante o período de conferência, o sistema deverá voltar a estar disponível em um tempo t, tal que t $\leq$ 2 minutos .

\textbullet\hspace{1mm} Probabilidade corromper dados/falha

Os dados salvos não devem ser corrompidos por uma falha.


\section{Evolução de sistema}
Atualmente, o sistema gerenciador de conferências tecnológicas é desenvolvido para conferências nacionais, e somente possui suporte à língua portuguesa. O pagamento de inscrição somente pode ser feito através da moeda local (Real).
Dentre as mudanças previstas tem-se:

\begin{description}

\item[Suporte à conferências internacionais:]  O sistema poderá oferecer suporte a conferências internacionais, com pagamento em moeda estrangeira. Para isso será necessário uma adaptação no mecanismo de venda, pagamento e um suporte a diversos idiomas.

\item[Vendas de produto:]  É previsto a inclusão de uma nova função relacionada à vendas de camisetas, fotos e DVD da conferência, sendo que o lucro obtido será repassado à instituições de caridade.

\item[Aumento desempenho:]  Com o aumento do número de acessos simultâneos, o desempenho esperado e requerido nesse documento deverá continuar o mesmo. 
\end{description}



%% Análise de riscos%%

\section{Análise de riscos}

O primeiro passo, para diminuir os ricos do projeto, a fim de evitar falhas, é identificá-los. Portanto, foi realizado uma avaliação de sua probabilidade e seriedade  (tabela \ref{r1} ). Posteriormente , foi necessário fazer um planejamento para os riscos (tabela \ref{r2} ).
 

\begin{table}[h!]
\begin{tabular}{clll}
\hline 
Tipo de risco & Riscos possíveis & Probabilidade & Efeitos\tabularnewline
\hline
\hline 
\multirow{2}{*}{Tecnologia}
 & O número de transações processadas por segundo suportado & Baixa & Sérios\tabularnewline
 & pelo banco de dados no sistema é insuficiente. &  & \tabularnewline
\multirow{2}{*}{Pessoal} & O treinamento necessário para o pessoal não está disponível. & Baixa & Sérios\tabularnewline
 & A pessoa que possuem conhecimento específico não está disponível. & Média & Sérios\tabularnewline
\multirow{2}{*}{Requisitos} & Mudanças de requisitos que requerem retrabalho & Alta & Catastróficos\tabularnewline
 &  maior do projeto são propostas. &  & \tabularnewline
\multirow{2}{*}{Estimativas} & O prazo necessário para desenvolver o software foi subestimado. & Alta & Sérios\tabularnewline
 & O tamanho do software foi subestimado. & Alta & Toleráveis\tabularnewline
\hline
\end{tabular}

\caption{Análise de riscos}
\label{r1}
\end{table}
%
\begin{table}[h!]
\begin{tabular}{ll}
\hline 
Risco & Estratégia\tabularnewline
\hline
\hline 
Indisponibilidade de pessoal & As pessoas da equipe devem compreender as tarefas uns dos outros.\tabularnewline
Prazo de desenvolvimento subestimado & Verificar a compra de componentes.\tabularnewline
Desempenho banco de dados & Verificar a possibilidade de usar um banco de dados melhor.\tabularnewline
\hline
\end{tabular}

\caption{Estratégia de gerenciamento de risco}
\label{r2}

\end{table}


\section{Glossário}
\begin{description}
\item[Requisitos funcionais] 
Definem as funções que o sistema deverá realizar para atender ao propósito pelo qual foi projetado .
\item [Requisitos não funcionais]
Definem aspectos que não são funcionalidades do sistema mas que são necessárias para o seu funcionamento
correto, como segurança, confiabilidade, desempenho, entre outros.

\item[Conferência Tecnológica] É um evento que dura um certo período e consiste de uma ou mais sessões (apresentações, palestras, minicursos) sobre uma ou mais determinadas áreas tecnológicas, como inovação em uma empresa para o desenvolvimento sustentável.

\item[Paper] Paper ou artigo científico é um trabalho acadêmico que apresenta resultado sucinto de pesquisa realizada de acordo com a metodologia de ciência aceita por uma comunidade de pesquisadores. Considera-se científico o artigo que foi submetido ao exame de outros cientistas, que verificam as informações, os métodos e a precisão lógico-metodológica das conclusões ou resultados obtidos. Pode ser resultado de sínteses de trabalhos maiores ou elaborados em número de três ou quatro, em substituição às teses e dissertações; são desenvolvidos, nesses casos, sob a assistência de um orientador acadêmico.

\item[Workshop] É uma reunião de grupos de trabalho interessados em determinado projeto ou atividade para discussão e/ou apresentação prática do referido projeto ou atividade.
\end{description}

\vspace{-12mm}
\renewcommand{\refname}{\section{Referências}}
  \begin{thebibliography}{2}  
\vspace{-5mm}
         \bibitem{RP}Ian Sommerville  \newblock\emph{Software engineering}.\newblock Addison-Wesley, 2001. 
      \bibitem{ULK} Roger S. Pressman \newblock\emph{Software engineering}.\newblock McGraw Hill Higher Education, 2000.
  \end{thebibliography}  
\end{document}
